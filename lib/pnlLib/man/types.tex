\section{Objects}

\subsection{The top-level object}

The PnlObject structure is used to simulate some inheritance between the
ojbects of Pnl.  It must be the first element of all the objects existing in
Pnl so that casting any object to a PnlObject is legal

\describestruct{PnlObject}
\begin{verbatim}
typedef unsigned int PnlType; 

typedef void (DestroyFunc) (void **);
typedef PnlObject* (CopyFunc) (PnlObject *);
typedef PnlObject* (NewFunc) (PnlObject *);
typedef void (CloneFunc) (PnlObject *dest, const PnlObject *src);
struct _PnlObject
{
  PnlType type; /*!< a unique integer id */
  const char *label; /*!< a string identifier (for the moment not useful) */
  PnlType parent_type; /*!< the identifier of the parent object is any,
                          otherwise parent_type=id */
  int nref; /*!< number of references on the object */ 
  DestroyFunc *destroy; /*!< frees an object */
  NewFunc     *constructor; /*!< New function */
  CopyFunc    *copy; /*!< Copy function */
  CloneFunc   *clone; /*!< Clone function */
};
\end{verbatim}

Here is the list of all the types actually defined
\begin{table}
  \centering
  \begin{tabular}{l|l}
    \hline
    PnlType & Description \\
    \hline
    PNL_TYPE_VECTOR & general vectors  \\
    PNL_TYPE_VECTOR_DOUBLE & real vectors \\
    PNL_TYPE_VECTOR_INT & integer vectors \\
    PNL_TYPE_VECTOR_COMPLEX & complex vectors \\
    PNL_TYPE_MATRIX & general matrices  \\
    PNL_TYPE_MATRIX_DOUBLE & real matrices \\
    PNL_TYPE_MATRIX_INT & integer matrices \\
    PNL_TYPE_MATRIX_COMPLEX & complex matrices \\
    PNL_TYPE_TRIDIAG_MATRIX & general tridiagonal matrices \\
    PNL_TYPE_TRIDIAG_MATRIX_DOUBLE & real  tridiagonal matrices \\
    PNL_TYPE_BAND_MATRIX & general band matrices \\
    PNL_TYPE_BAND_MATRIX_DOUBLE & real band matrices \\
    PNL_TYPE_SP_MATRIX & sparse general matrices  \\
    PNL_TYPE_SP_MATRIX_DOUBLE & sparse real matrices \\
    PNL_TYPE_SP_MATRIX_INT & sparse integer matrices \\
    PNL_TYPE_SP_MATRIX_COMPLEX & sparse complex matrices \\
    PNL_TYPE_HMATRIX & general hyper matrices \\
    PNL_TYPE_HMATRIX_DOUBLE & real hyper matrices \\
    PNL_TYPE_HMATRIX_INT & integer hyper matrices \\
    PNL_TYPE_HMATRIX_COMPLEX & complex hyper matrices \\
    PNL_TYPE_BASIS & bases \\
    PNL_TYPE_RNG & random number generators \\
    PNL_TYPE_LIST & doubly linked list \\
    PNL_TYPE_ARRAY & array
  \end{tabular}
  \caption{PnlTypes}
  \label{types}
\end{table}

We provide several macros for manipulating PnlObejcts.
\begin{itemize}
\item \describemacro{PNL_OBJECT}{o}
  \sshortdescribe Cast any object into a PnlObject

\item \describemacro{PNL_VECT_OBJECT}{o}
  \sshortdescribe Cast any object into a PnlVectObject

\item \describemacro{PNL_MAT_OBJECT}{o}
  \sshortdescribe Cast any object into a PnlMatObject

\item \describemacro{PNL_SP_MAT_OBJECT}{o}
  \sshortdescribe Cast any object into a PnlSpMatObject

\item \describemacro{PNL_HMAT_OBJECT}{o}
  \sshortdescribe Cast any object into a PnlHmatObject

\item \describemacro{PNL_BAND_MAT_OBJECT}{o}
  \sshortdescribe Cast any object into a PnlBandMatObject

\item \describemacro{PNL_TRIDIAGMAT_OBJECT}{o}
  \sshortdescribe Cast any object into a PnlTridiagMatObject

\item \describemacro{PNL_BASIS_OBJECT}{o}
  \sshortdescribe Cast any object into a PnlBasis

\item \describemacro{PNL_RNG_OBJECT}{o}
  \sshortdescribe Cast any object into a PnlRng

\item \describemacro{PNL_LIST_OBJECT}{o}
  \sshortdescribe Cast any object into a PnlList

\item \describemacro{PNL_LIST_ARRAY}{o}
  \sshortdescribe Cast any object into a PnlArray

\item \describemacro{PNL_GET_TYPENAME}{o}
  \sshortdescribe Return the name of the type of any object inheriting from PnlObject

\item \describemacro{PNL_GET_TYPE}{o}
  \sshortdescribe Return the type of any object inheriting from PnlObject
  
\item \describemacro{PNL_GET_PARENT_TYPE}{o}
  \sshortdescribe Return the parent type of any object inheriting from PnlObject
\end{itemize}

\begin{itemize}
\item \describefun{\PnlObject\ptr }{pnl_object_create}{PnlType t}
  \sshortdescribe Create an empty PnlObject of type \var{t} which can any of
  the registered types, see Table~\ref{types}.
\end{itemize}

\subsection{List object}

This section describes functions for creating an manipulating lists. Lists are
internally stored as doubly linked lists.

The structures and functions related to lists are declared in
\verb!pnl/pnl_list.h!.

\describestruct{PnlList}\describestruct{PnlCell}
\begin{verbatim}
typedef struct _PnlCell PnlCell;
struct _PnlCell
{
  struct _PnlCell *prev;  /*!< previous cell or 0 */
  struct _PnlCell *next;  /*!< next cell or 0 */
  PnlObject *self;       /*!< stored object */
};


typedef struct _PnlList PnlList;
struct _PnlList
{
  /**
   * Must be the first element in order for the object mechanism to work
   * properly. This allows any PnlList pointer to be cast to a PnlObject
   */
  PnlObject object; 
  PnlCell *first; /*!< first element of the list */
  PnlCell *last; /*!< last element of the list */
  PnlCell *curcell; /*!< last accessed element,
                         if never accessed is NULL */
  int icurcell; /*!< index of the last accessed element,
                     if never accessed is NULLINT */
  int len; /*!< length of the list */
};
\end{verbatim}

\textbf{Important note}: Lists only store addresses of objects. So when an
object is inserted into a list, only its address is stored into the list. This
implies that you \textbf{must not} free any objects inserted into a list. The
deallocation is automatically handled by the function \reffun{pnl_list_free}.

\begin{itemize}
\item \describefun{\PnlList \ptr }{pnl_list_new}{}
  \sshortdescribe Create an empty list
\item \describefun{\PnlCell \ptr }{pnl_cell_new}{}
  \sshortdescribe Create an cell list
\item \describefun{\PnlList\ptr }{pnl_list_copy}{const \PnlList\ptr A}
  \sshortdescribe Create a copy of a \PnlList. Each element of the
  list \var{A} is copied by calling the its copy member.
\item \describefun{void}{pnl_list_clone}{\PnlList \ptr dest, const
  \PnlList\ptr src}
  \sshortdescribe Copy the content of \var{src} into the already existing
  list \var{dest}. The list \var{dest} is automatically resized. This is a
  hard copy, the contents of both lists are independent after cloning.
\item \describefun{void}{pnl_list_free}{\PnlList  \ptr \ptr L}
  \sshortdescribe Free a list
\item \describefun{void}{pnl_cell_free}{\PnlCell  \ptr \ptr c}
  \sshortdescribe Free a list
\item \describefun{\PnlObject\ptr}{pnl_list_get}{
    \PnlList \ptr L, int i}
  \sshortdescribe This function returns the content of the \var{i}--th cell of
  the list \var{L}. This function is optimized for linearly accessing all the
  elements, so it can be used inside a for loop for instance.
\item \describefun{void}{pnl_list_insert_first}{\PnlList  \ptr L,
    \PnlObject  \ptr o}
  \sshortdescribe Insert the object \var{o} on top of the list \var{L}. Note that
  \var{o} is not copied in \var{L}, so do  {\bf not} free \var{o} yourself, it
  will be done automatically when calling \reffun{pnl_list_free}
\item \describefun{void}{pnl_list_insert_last}{\PnlList  \ptr L,
    \PnlObject  \ptr o}
  \sshortdescribe Insert the object \var{o} at the bottom of the list \var{L}. Note that
  \var{o} is not copied in \var{L}, so do  {\bf not} free \var{o} yourself, it
  will be done automatically when calling \reffun{pnl_list_free}
\item \describefun{void}{pnl_list_remove_last}{\PnlList  \ptr L}
  \sshortdescribe Remove the last element of the list \var{L} and frees it.
\item \describefun{void}{pnl_list_remove_first}{\PnlList  \ptr L}
  \sshortdescribe Remove the first element of the list \var{L} and frees it.
\item \describefun{void}{pnl_list_remove_i}{\PnlList  \ptr L, int i}
  \sshortdescribe Remove the \var{i-th} element of the list \var{L} and frees it.
\item \describefun{void}{pnl_list_concat}{\PnlList  \ptr L1,
    \PnlList  \ptr L2}
  \sshortdescribe Concatenate the two lists \var{L1} and \var{L2}. The
  resulting list is store in \var{L1} on exit. Do {\bf not} free \var{L2}
  since concatenation does not actually copy objects but only manipulates
  addresses.
\item \describefun{void}{pnl_list_resize}{\PnlList \ptr L, int n}
  \sshortdescribe Change the length of \var{L} to become \var{n}. If the length
  of \var{L} id increased, the extra elements are set to NULL.
\item \describefun{void}{pnl_list_print}{const \PnlList  \ptr L}
  \sshortdescribe Only prints the types of each element. When  the
  \PnlObject object has a print member, we will use it.
\end{itemize}

\subsection{Array object}

This section describes functions for creating and manipulating arrays of
PnlObjects.

The structures and functions related to arrays are declared in
\verb!pnl/pnl_array.h!.

\describestruct{PnlArray}
\begin{verbatim}
typedef struct _PnlArray PnlArray;
struct _PnlArray
{
  /**
   * Must be the first element in order for the object mechanism to work
   * properly. This allows any PnlArray pointer to be cast to a PnlObject
   */
  PnlObject object; 
  int size;
  PnlObject **array;
  int mem_size;
};
\end{verbatim}

\textbf{Important note}: Arrays only store addresses of objects. So when an
object is inserted into an array, only its address is stored into the array. This
implies that you \textbf{must not} free any objects inserted into a array. The
deallocation is automatically handled by the function \reffun{pnl_array_free}.

\begin{itemize}
\item \describefun{\PnlArray \ptr }{pnl_array_new}{}
  \sshortdescribe Create an empty array
\item \describefun{\PnlArray \ptr }{pnl_array_create}{int n}
  \sshortdescribe Create an array of length \var{n}.
\item \describefun{\PnlArray\ptr }{pnl_array_copy}{const \PnlArray\ptr A}
  \sshortdescribe Create a copy of a \PnlArray. Each element of the
  array \var{A} is copied by calling the \var{A[i].object.copy}.
\item \describefun{void}{pnl_array_clone}{\PnlArray \ptr dest, const
  \PnlArray\ptr src}
  \sshortdescribe Copy the content of \var{src} into the already existing
  array \var{dest}. The array \var{dest} is automatically resized. This is a
  hard copy, the contents of both arrays are independent after cloning.
\item \describefun{void}{pnl_array_free}{\PnlArray  \ptr \ptr}
  \sshortdescribe Free an array and all the objects hold by the array.
\item \describefun{int}{pnl_array_resize}{\PnlArray \ptr  T, int size}
  \sshortdescribe Resize \var{T} to be \var{size} long. As much as possible of
  the original data is kept.
\item \describefun{\PnlObject\ptr}{pnl_array_get}{
    \PnlArray \ptr T, int i}
  \sshortdescribe This function returns the content of the \var{i}--th cell of
  the array \var{T}. No copy is made.
\item \describefun{\PnlObject\ptr}{pnl_array_set}{
    \PnlArray \ptr T, int i, \PnlObject\ptr O}
  \sshortdescribe \var{T[i] = O}. No copy is made, so the object \var{O} must
  not be freed manually.
\item \describefun{void}{pnl_array_print}{\PnlArray  \ptr}
  \sshortdescribe Not yet implemented because it would require that the
  structure \PnlObject has a field copy.
\end{itemize}


%%% Local Variables: 
%%% mode: latex
%%% TeX-master: "pnl-manual"
%%% End: 
